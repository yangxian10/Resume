% LaTeX Curriculum Vitae Template
%
% Copyright (C) 2004-2009 Jason Blevins <jrblevin@sdf.lonestar.org>
% http://jblevins.org/projects/cv-template/
%
% You may use use this document as a template to create your own CV
% and you may redistribute the source code freely. No attribution is
% required in any resulting documents. I do ask that you please leave
% this notice and the above URL in the source code if you choose to
% redistribute this file.

\documentclass[a4paper]{article}
\usepackage[BoldFont,SlantFont]{xeCJK}
\usepackage{xcolor}
\setmainfont{Times New Roman}
\setCJKmainfont[BoldFont={STXihei},ItalicFont={KaiTi}]{SimSun}
\usepackage{hyperref}
\usepackage{geometry}
\newcommand{\wuhao}{\fontsize{11pt}{\baselineskip}\selectfont}

% Comment the following lines to use the default Computer Modern font
% instead of the Palatino font provided by the mathpazo package.
% Remove the 'osf' bit if you don't like the old style figures.
%\usepackage[T1]{fontenc}
%\usepackage[sc,osf]{mathpazo}

% Set your name here
\def\name{姜波}

% Replace this with a link to your CV if you like, or set it empty
% (as in \def\footerlink{}) to remove the link in the footer:
\def\footerlink{http://www.geniusy.com/jiangbo-cv-chinese.pdf}

% The following metadata will show up in the PDF properties
\hypersetup{
  colorlinks = true,
  urlcolor = black,
  pdfauthor = {\name},
  pdfkeywords = {金融, 经济, 统计},
  pdftitle = {\name: 简历},
  pdfsubject = {简历},
  pdfpagemode = UseNone
}

\geometry{
  body={16cm, 24cm},
  left=2.5cm,
  top=2cm
}

% Customize page headers
\usepackage{fancyhdr}
\pagestyle{fancy}
\fancyhead[C]{\includegraphics[scale=0.6]{logo2.jpg}}
\renewcommand{\headrulewidth}{0pt}
\fancyfoot[C]{\thepage}
\headheight=2cm

% Custom section fonts
\usepackage{sectsty}
\sectionfont{\rmfamily\mdseries\Large}
\subsectionfont{\rmfamily\mdseries\itshape\large}

% Other possible font commands include:
% \ttfamily for teletype,
% \sffamily for sans serif,
% \bfseries for bold,
% \scshape for small caps,
% \normalsize, \large, \Large, \LARGE sizes.

% Don't indent paragraphs.
\setlength{\parindent}{0em}

% Make lists without bullets
\renewenvironment{itemize}{
  \begin{list}{}{
    \setlength{\leftmargin}{1.5em}
    \setlength{\itemsep}{0pt}
  }
}{
  \end{list}
}

\renewcommand\arraystretch{1.35}

\begin{document}

% Place name at left
{\huge \name}

% Alternatively, print name centered and bold:
%\centerline{\huge \bf \name}

\vspace{0.25in}
\begin{minipage}{0.55\linewidth}
  \href{http://www.swufe.edu.cn/}{西南财经大学} \\
  金融学院 \\
  四川省成都市温江区柳台大道 611130
\end{minipage}
\vspace{0.25in}
\begin{minipage}{0.40\linewidth}
  手机: ($+$86)186-2800-0989 \\
  邮箱: \href{mailto:geniusy.jiang@gmail.com}{\tt geniusy.jiang@gmail.com} \\
  主页: \href{http://www.geniusy.com/}{\tt http://www.geniusy.com/}
\end{minipage}

\vspace{-1.5em}
\section*{个人信息}
\begin{itemize}
\item 性别: 男\hspace{1em}年龄: 26\hspace{1em}籍贯: 重庆
\end{itemize}

\section*{教育信息}
\begin{itemize}
  \item 西南财经大学金融学院,金融学,硕博连读,2010--至今
  \item 南京审计学院国际审计学院,审计学(ACCA方向),本科,2005--2009
\end{itemize}

\section*{研究方向}
\begin{itemize}
\item 公司金融、资本市场
\end{itemize}

\section*{教学经历}
\begin{itemize}
\item 讲师,{\it 统计学概论}(西南财经大学成人教育学院专科),2011年12月
\item 助教,{\it 公司金融}(三年级本科,全英文),2011秋期
\item 助教,{\it 管理学原理}(二年级本科,全英文),2011年春期
\item 助教,{\it 会计学}(二年级本科,全英文),2010秋期
\end{itemize}

\section*{工作经历}
\begin{itemize}
\item 2011/07--2011/09:中国家庭金融调查广东组(人数最多的组)督导兼领队
\item 2010/01--2010/03:武汉富达天宇投资咨询有限公司实习,任助理黄金分析师
\item 2008/07--2008/08:爱楷企业管理咨询有限公司(即ACCA上海代表处)公关部实习
\end{itemize}

\section*{社会工作}
\begin{itemize}
\item 2006/10--2011/10:ACCA学习网(52ACCA.com)创始人兼站长
\item 2008/01--2008/07:南京审计学院学生会新闻宣传中心主任
\item 2007/08--2008/01:南京审计学院学生会网络工作室部长
\item 2006/06--2007/07:南京审计学院网络管理中心网络协管小组成员
\item 2006/06--2007/07:南京审计学院团委宣传部网站工作室组长
\item 2006/04--2007/04:南京审计学院国际审计学院学生会网络技术部部长
\item 2005/10--2006/07:南京审计学院计算机协会财务部部长
\end{itemize}

\newpage
\section*{资格与证书}
\vspace{-1em}
\begin{table}[!htbp]
\begin{tabular}[c]{ll}
\hspace{0.85em}ACCA(英国公认注册会计师,英文):&\qquad 通过10门 (共14门)\\
\hspace{0.85em}CISA(国际注册信息系统审计师,英文版):&\qquad 562 (450分通过)\\
\hspace{0.85em}Bloomberg Assessment Test(BAT):&\qquad 76.13 (中国及全球排名前3\%)\\
\hspace{0.85em}CET-6:&\qquad 601分\\
\hspace{0.85em}CET-4:&\qquad 609分\\
\hspace{0.85em}国家计算机等级考试三级-网络技术:&\qquad 合格\\
\hspace{0.85em}江苏省计算机等级考试二级-VFP:&\qquad 优秀\\
\end{tabular}
\end{table}
\vspace{-1.5em}

\section*{电脑技能}
\subsection*{编程语言:}
\begin{itemize}
\item {\bfseries 程序设计}:C(良好),Visual Basic(良好),Pascal(基础)
\item {\bfseries 网页设计}:HTML+CSS(熟练),PHP(基础),ASP(基础)
\end{itemize}

\subsection*{应用软件:}
\begin{itemize}
\item {\bfseries 统计及计算软件}:SAS(熟练),Stata(熟练),Eviews(熟练),Matlab(基础),R(基础)
\item {\bfseries 数据库系统}:SQL Server(良好),Visual FoxPro(良好),MySQL(基础),Access(良好)
\item {\bfseries 图像处理}:Photoshop (基础), Flash (基础)
\item {\bfseries 办公与排版软件}:\LaTeX (熟练),Word(熟练),Excel(熟练),PowerPoint(熟练)
\end{itemize}

\subsection*{操作系统:}
\begin{itemize}
\item {\bfseries Windows}:1998年开始使用,熟练操作每个版本的桌面版和服务器版
\item {\bfseries Mac OS}:2008年后主要操作系统
\item {\bfseries Linux}:熟练操作多个发行版的桌面版和服务器版
\end{itemize}

\section*{荣誉与奖励:}
\begin{itemize}
\item 西南财经大学2011--2012学年研究生一等学术奖学金
\item 西南财经大学2010--2011学年优秀研究生奖学金
\item 西南财经大学2010--2011学年研究生二等学术奖学金
\item 南京审计学院2007--2008学年ACCA优秀学生奖学金三等奖
\item 南京审计学院2007--2008学年校优秀学生会干部
\item 南京审计学院2006--2007学年ACCA优秀学生奖学金二等奖
\item 南京审计学院2005--2006学年校优秀学生奖学金三等奖
\end{itemize}

\bigskip

% Footer
\begin{center}
  \begin{footnotesize}
    Last updated: \today \\
    \href{\footerlink}{\texttt{\footerlink}}
  \end{footnotesize}
\end{center}

\end{document}
